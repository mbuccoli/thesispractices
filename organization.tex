\chapter{Organization of a chapter}

In this Chapter we provide some useful advices on how to organize and present the information within a chapter, from the definition of topics to their actual writing. 

A first advice is actually given by the first sentence: write down, in the first sentence of the chapter, what is its goal, so that it is not obvious (of course the state of the art will provide the state of the art). 

When defining the topics, choose what you are going to say and organize a structure of your chapter into sections and possibly subsection. For each section, write down one or two sentences explaining (for you) what you would like to state. In the best scenario, the reading all the sentences should provide a sense of flow, of reasoning, around the topic. So, try to imagine how you want to switch from one Section to another. A common way is to describe the topic of the Section and then highlight its limitation on issues, that are addressed by the next topic. For example:

\textbf{X. Approaches}
In this section we describe the approaches. We can roughly divide the approaches into two categories: context-based, which use user-generated data, and content-based, which use automatically-extracted data.

\textbf{X.1 Context-based approaches}\\
Context-based approaches rely on user-generated data to build models. \\
This requires users to generate/annotate data, which is rather cognitively heavy and time consuming. Content-based approaches address this issue by using automatically-extracted data to build models.

\textbf{X.2 Content-based approaches}\\
...

